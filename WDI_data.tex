\documentclass[a4paper]{article}
\setlength{\parindent}{0pt}

\begin{document}


\section{GDP (current LCU)}
GDP at purchaser's prices is the sum of gross value added by all resident producers in the economy plus any product taxes and minus any subsidies not included in the value of the products. It is calculated without making deductions for depreciation of fabricated assets or for depletion and degradation of natural resources. Data are in current local currency.

ID: NY.GDP.MKTP.CN

Source: World Bank national accounts data, and OECD National Accounts data files.

License:  CC BY-4.0 

Long Definition: GDP at purchaser's prices is the sum of gross value added by all resident producers in the economy plus any product taxes and minus any subsidies not included in the value of the products. It is calculated without making deductions for depreciation of fabricated assets or for depletion and degradation of natural resources. Data are in current local currency.

Periodicity: Annual

Topic: Economic Policy \& Debt: National accounts: Local currency at current prices: Aggregate indicators

\section{General government final consumption expenditure (current LCU)}

General government final consumption expenditure (formerly general government consumption) includes all government current expenditures for purchases of goods and services (including compensation of employees). It also includes most expenditures on national defense and security, but excludes government military expenditures that are part of government capital formation. Data are in current local currency.

ID: NE.CON.GOVT.CN

Source: World Bank national accounts data, and OECD National Accounts data files.

License:  CC BY-4.0 

Long Definition: General government final consumption expenditure (formerly general government consumption) includes all government current expenditures for purchases of goods and services (including compensation of employees). It also includes most expenditures on national defense and security, but excludes government military expenditures that are part of government capital formation. Data are in current local currency.

Periodicity: Annual

Topic: Economic Policy \& Debt: National accounts: Local currency at current prices: Expenditure on GDP

\section{Inflation, consumer prices (annual \%)}
Inflation as measured by the consumer price index reflects the annual percentage change in the cost to the average consumer of acquiring a basket of goods and services that may be fixed or changed at specified intervals, such as yearly. The Laspeyres formula is generally used.

ID: FP.CPI.TOTL.ZG

Source: International Monetary Fund, International Financial Statistics and data files.

License:  CC BY-4.0 

Aggregation Method: Median

Long Definition: Inflation as measured by the consumer price index reflects the annual percentage change in the cost to the average consumer of acquiring a basket of goods and services that may be fixed or changed at specified intervals, such as yearly. The Laspeyres formula is generally used.

Periodicity: Annual

Topic: Financial Sector: Exchange rates \& prices

\section{Industry (including construction), value added (current LCU)}
Industry (including construction) corresponds to ISIC divisions 05-43 and includes manufacturing (ISIC divisions 10-33). It comprises value added in mining, manufacturing (also reported as a separate subgroup), construction, electricity, water, and gas. Value added is the net output of a sector after adding up all outputs and subtracting intermediate inputs. It is calculated without making deductions for depreciation of fabricated assets or depletion and degradation of natural resources. The origin of value added is determined by the International Standard Industrial Classification (ISIC), revision 4. Data are in current local currency.

ID: NV.IND.TOTL.CN

Source: World Bank national accounts data, and OECD National Accounts data files.

License:  CC BY-4.0 

General Comments: Note: Data for OECD countries are based on ISIC, revision 4.

Long Definition: Industry (including construction) corresponds to ISIC divisions 05-43 and includes manufacturing (ISIC divisions 10-33). It comprises value added in mining, manufacturing (also reported as a separate subgroup), construction, electricity, water, and gas. Value added is the net output of a sector after adding up all outputs and subtracting intermediate inputs. It is calculated without making deductions for depreciation of fabricated assets or depletion and degradation of natural resources. The origin of value added is determined by the International Standard Industrial Classification (ISIC), revision 4. Data are in current local currency.

Periodicity: Annual

Topic: Economic Policy \& Debt: National accounts: Local currency at current prices: Value added

\section{Unemployment, total (\% of total labor force) (modeled ILO estimate)}

Unemployment refers to the share of the labor force that is without work but available for and seeking employment.

ID: SL.UEM.TOTL.ZS

Source: International Labour Organization. “ILO Modelled Estimates and Projections database ( ILOEST )” ILOSTAT. Accessed January 07, 2025. ilostat.ilo.org/data.

License:  CC BY-4.0 

Aggregation Method: Weighted average

Development Relevance: Paradoxically, low unemployment rates can disguise substantial poverty in a country, while high unemployment rates can occur in countries with a high level of economic development and low rates of poverty. In countries without unemployment or welfare benefits people eke out a living in vulnerable employment. In countries with well-developed safety nets workers can afford to wait for suitable or desirable jobs. But high and sustained unemployment indicates serious inefficiencies in resource allocation. Youth unemployment is an important policy issue for many economies. Young men and women today face increasing uncertainty in their hopes of undergoing a satisfactory transition in the labour market, and this uncertainty and disillusionment can, in turn, have damaging effects on individuals, communities, economies and society at large. Unemployed or underemployed youth are less able to contribute effectively to national development and have fewer opportunities to exercise their rights as citizens. They have less to spend as consumers, less to invest as savers and often have no "voice" to bring about change in their lives and communities. Widespread youth unemployment and underemployment also prevents companies and countries from innovating and developing competitive advantages based on human capital investment, thus undermining future prospects. Unemployment is a key measure to monitor whether a country is on track to achieve the Sustainable Development Goal of promoting sustained, inclusive and sustainable economic growth, full and productive employment and decent work for all. [SDG Indicator 8.5.2]

General Comments: National estimates are also available in the WDI database. Caution should be used when comparing ILO estimates with national estimates.

Limitations and Exceptions: The criteria for people considered to be seeking work, and the treatment of people temporarily laid off or seeking work for the first time, vary across countries. In many cases it is especially difficult to measure employment and unemployment in agriculture. The timing of a survey can maximize the effects of seasonal unemployment in agriculture. And informal sector employment is difficult to quantify where informal activities are not tracked. There may be also persons not currently in the labour market who want to work but do not actively "seek" work because they view job opportunities as limited, or because they have restricted labour mobility, or face discrimination, or structural, social or cultural barriers. The exclusion of people who want to work but are not seeking work (often called the "hidden unemployed" or "discouraged workers") is a criterion that will affect the unemployment count of both women and men. However, women tend to be excluded from the count for various reasons. Women suffer more from discrimination and from structural, social, and cultural barriers that impede them from seeking work. Also, women are often responsible for the care of children and the elderly and for household affairs. They may not be available for work during the short reference period, as they need to make arrangements before starting work. Further, women are considered to be employed when they are working part-time or in temporary jobs, despite the instability of these jobs or their active search for more secure employment.

Long Definition: Unemployment refers to the share of the labor force that is without work but available for and seeking employment.

Notesfromoriginalsource: Given the exceptional situation, including the scarcity of relevant data, the ILO modeled estimates and projections from 2020 onwards are subject to substantial uncertainty.

Periodicity: Annual

Statistical Concept and Methodology: The standard definition of unemployed persons is those individuals without work, seeking work in a recent past period, and currently available for work, including people who have lost their jobs or voluntarily left work. In addition, persons who did not look for work but have an arrangement for a future job are also counted as unemployed. Still, some unemployment is unavoidable—at any time, some workers are temporarily unemployed between jobs as employers look for the right workers and workers search for better jobs. The labor force or the economically active portion of the population serves as the base for this indicator, not the total population. The series is part of the "ILO modeled estimates database," including nationally reported observations and imputed data for countries with missing data, primarily to capture regional and global trends with consistent country coverage. Country-reported microdata is based mainly on nationally representative labor force surveys, with other sources (e.g., household surveys and population censuses) considering differences in the data source, the scope of coverage, methodology, and other country-specific factors. Country analysis requires caution where limited nationally reported data are available. A series of models are also applied to impute missing observations and make projections. However, imputed observations are not based on national data, are subject to high uncertainty, and should not be used for country comparisons or rankings. For more information: ilostat.ilo.org/resources/concepts-and-definitions/ilo-modelled-estimates

Topic: Social Protection \& Labor: Unemployment

\section{Gross fixed capital formation (current LCU)}
Gross fixed capital formation (formerly gross domestic fixed investment) includes land improvements (fences, ditches, drains, and so on); plant, machinery, and equipment purchases; and the construction of roads, railways, and the like, including schools, offices, hospitals, private residential dwellings, and commercial and industrial buildings. According to the 1993 SNA, net acquisitions of valuables are also considered capital formation. Data are in current local currency.

ID: NE.GDI.FTOT.CN

Source: World Bank national accounts data, and OECD National Accounts data files.

License:  CC BY-4.0 

Long Definition: Gross fixed capital formation (formerly gross domestic fixed investment) includes land improvements (fences, ditches, drains, and so on); plant, machinery, and equipment purchases; and the construction of roads, railways, and the like, including schools, offices, hospitals, private residential dwellings, and commercial and industrial buildings. According to the 1993 SNA, net acquisitions of valuables are also considered capital formation. Data are in current local currency.

Periodicity: Annual

Topic: Economic Policy \& Debt: National accounts: Local currency at current prices: Expenditure on GDP

\end{document}